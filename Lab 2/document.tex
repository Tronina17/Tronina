\documentclass{beamer}
\usetheme{Boadilla}

\usepackage[utf8]{inputenc}
\usepackage[russian]{babel}

\usepackage{xcolor}


\title{Введение в специальность}
\subtitle{Криптография на эллиптических кривых}
\author{Тронина София Александровна}
\setbeamercolor{author}{fg=blue}
\institute{"Компьютерная безопасность" 1 курс\\
Балтийский Федеральный Университет им. И. Канта}
\date{2020}

\begin{document}
	\begin{frame}
		\titlepage
	\end{frame}

	\begin{frame}
		\section{{\color{violet}Задачи:}}
			\subsection{Изучить алгоритм маркировки единичного сообщения.}
			\subsection{Изучить процедуру демаркировки.}
			\subsection{Изучить аналог системы Диффи-Хеллмана.}
			\subsection{Разобрать все на примерах.}
		\section{{\color{violet}Методы решения:}}
			\subsection{Изучение литературы.}
			\subsection{Расчеты и измерения.}
		\tableofcontents	
	\end{frame}

	\begin{frame}
		\frametitle{{\color{violet}Результаты:}}
		\begin{enumerate}
			\item В ходе работы я научилась маркировать и демаркировать еденичные сообщения на эллиптических кривых.
			\item Кроме того выяснила, что  вероятность неуспешной маркировки за $\kappa$ попыток $ \sim \left( \frac{1}{2}\right) ^{\kappa} = \frac{1}{2^{\kappa}} $ 
			\item А вероятность успешной маркировки (за $\kappa$ попыток) $ \sim 1-\frac{1}{2^{\kappa}} \approx 0,999$ (при $\kappa = 50). $ 
		\end{enumerate}	
	\end{frame}
\end{document}